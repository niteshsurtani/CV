%\title{My two column CV}
%
% tccv (two columns curriculum vitae) is a LaTeX class inspired by
% the template found at latextemplates.com by Alessandro Plasmati.
%
% Create by Nicola Fontana, the original files can be downloaded from:
% http://dev.entidi.com/p/tccv/
%
\documentclass{tccv}

\usepackage[english]{babel}
\usepackage{ulem}
\usepackage[object=vectorian]{pgfornament}
\usepackage{setspace}
\usepackage{multirow}
\usepackage{hhline}
\usepackage{fancyhdr}
\usepackage{array}

\fancyhf{}
\fancyfoot[L]{Pooria Azimi - Curriculum vit\ae}
\fancyfoot[C]{(\thepage)}
\fancyfoot[R]{\href{http://bit.ly/pooria-azimi-cv}{http://bit.ly/pooria-azimi-cv}}
\renewcommand\headrulewidth{0pt}
\renewcommand\footrulewidth{1pt}
\pagestyle{fancy}

  
  
  
\begin{document}

\thispagestyle{empty}
\part{Pooria Azimi}




\section{Research Interests}


\begin{research_interest}

\item{Database and IR Systems}
     {}

\item{Operating Systems}
     {File systems - IPC - Microkernels}

\item{Parallel System Programming}
     {the "Actor Model"}

\item{Human-Computer Interaction}
     {}

\end{research_interest}








\section{Education}

B.Sc. in {\bf Computer Science} (2009 -- Present)
\\[1.5pt]
{\bf \href{https://en.wikipedia.org/wiki/Amirkabir_University_of_Technology}{Amirkabir University of Technology}} (Iran)
\\[1.7pt]
GPA (last 60 CS credits): {\bf 16.5}/20
\bigskip\\
Ranked 590 (among 400,000+) in 2009 National Matriculation Exam ({\bf top \%0.2})








\section{Communication skills}


\begin{factlist}

\item{English}{Fluent}

\item{Persian}{Native speaker}

\item{German}{Basic understanding}

\end{factlist}


\sectionline


{\renewcommand{\arraystretch}{1.6}
\begin{table}[ph]
  \centering
	\begin{tabular}{C{3.5cm}|C{4cm}}
 	   {\sc TOEFL }i{\sc BT} & {\sc GRE General (\uline{expected}\textsuperscript{$*$})}\\[5pt]
	   \hline
        \vspace{-12pt}
 	   	\begin{tabular}{R{1.7cm}|L{0.9cm}}
			Reading & 30 \\
			Listening & 29\\
			Speaking & 22\textsuperscript{$**$} \\
			Writing & 28\\
			\hline
			{\bf Total} & {\bf 109}
		\end{tabular}
		&
 	   	\begin{tabular}{R{2.3cm}|L{0.8cm}} 
			Verbal Reasoning & 160+\\
			Quantitative Reasoning & 170\\
			Analythical Writing & 4.5+
		\end{tabular}
	\end{tabular}
\end{table}
{\renewcommand{\arraystretch}{1}


\textsuperscript{$*$} {\it Only the paper-based GRE test is administrated in Iran (with the results being made available on {\bf December 9\textsuperscript{th}}), but based on a few prep tests, I \uline{expect} to approximately score as stated above.}
\medskip\\
\textsuperscript{$**$} {\it I'm taking another iBT test (with the results being made available on {\bf November 12\textsuperscript{th}}) to improve my score in the Speaking section.}








\personal
    {365, E2, Ekbatan, Tehran, Iran}
    {+98 (935) 431 26 45}
    {pooriaazimi@gmail.com}





\section{Technical skills}

\begin{factlist}

\item{Languages}
     {
     \href{http://www.oracle.com/technetwork/java/}{Java} -- 
     \href{https://www.ruby-lang.org/en/}{Ruby} -- 
     \href{http://www.erlang.org}{Erlang} -- 
     \href{https://en.wikipedia.org/wiki/C_(programming_language)}{C} -- 
     \href{http://php.net}{PHP} -- 
     \href{http://www.scala-lang.org}{Scala} -- 
     \href{https://en.wikipedia.org/wiki/Objective-C}{Objective-C} -- 
     \href{http://coffeescript.org}{CoffeeScript}%
     }

\item{Databases}
     {
     \href{http://www.postgresql.org}{PostgreSQL} -- 
     \href{http://www.mongodb.org}{MongoDB} -- 
     \href{http://www.neo4j.org}{Neo4j} -- 
     \href{http://redis.io}{Redis} -- 
     \href{http://www.microsoft.com/en-us/sqlserver/default.aspx}{Microsoft SQL Server}
     }

\item{Web}
     {
     \href{http://nodejs.org}{Node.js} -- 
     \href{http://rubyonrails.org}{Ruby on Rails} -- 
     \href{http://php.net}{PHP} -- 
     \href{http://www.sinatrarb.com}{Sinatra} -- 
     \href{http://backbonejs.org}{Backbone.js} -- 
     \href{http://emberjs.com}{Ember.js}
     }

\item{Misc.}
     {
     \href{http://git-scm.com}{git} -- 
     \href{http://www.selenic.com/mercurial/}{hg} -- 
     \href{http://www.vim.org}{vi} -- 
     \href{http://awk.info}{awk} -- 
     \href{http://www.gnu.org/software/sed/}{sed} -- 
     \href{http://nginx.org/en/}{nginx} -- 
     \href{http://www.webkit.org}{WebKit} -- 
     \href{http://www.antlr.org}{ANTLR} -- 
     \href{http://www.latex-project.org}{\LaTeX}
     }

\end{factlist}




\vspace{8pt}





\section{Teaching Experiences}

\begin{teaching}

% Teacher Assistant

\item{Winter 2011}
	 {}
	 {Database Design}
	 {Teacher Assistant}


\item{Winter 2012}
	 {}
	 {Database Design}
	 {Teacher Assistant}


\item{Winter 2012}
	 {}
	 {Artificial Intelligence}
	 {Teacher Assistant}


\item{Winter 2013}
	 {}
	 {Database Design}
	 {Teacher Assistant}


% Presenter

%\item{Winter 2011}
%	 {Database Design Course}
%	 {Database Storage and Indices}
%	 {Presenter}


\item{Spring 2012}
	 {4th National Linux Festivel}
	 {Advanced Unix Command-line}
	 {Presenter}


\item{Spring 2012}
	 {AUT-CEIT Computing Festival}
	 {Introduction to Node.js}
	 {Presenter}


\item{Spring 2013}
	 {AUT Database Workshop}
	 {PostgreSQL vs. MySQL}
	 {Presenter}


\item{Spring 2013}
	 {AUT Database Workshop}
	 {Introduction to MongoDB}
	 {Presenter}


\item{Spring 2013}
	 {AUT Database Workshop}
	 {Basics of Neo4j and Redis}
	 {Presenter}

\end{teaching}










\section{Projects}

\begin{project_list}

\item{July 2011 -- April 2012}
     {Iran Telecommunications Research Center}
     {Kavandeh Search Engine}
	 {Improving link-based web page ranking algorithms in a Persian-only search engine, using various statistical and heurostic metods.}
     {
     	\href{http://nutch.apache.org}{Apache Nutch} -- 
		\href{http://lucene.apache.org/solr/}{Apache Solr} -- 
		\href{http://www.webkit.org}{WebKit} -- 
		\href{http://www.oracle.com/technetwork/java/}{Java} -- 
		\href{https://en.wikipedia.org/wiki/C\%2B\%2B}{C++}
	}

\vspace{5pt}

\item{March 2012 -- June 2012}
     {}
     {Visual WebPage Segmentation}
	 {Detecting web page structure using statistical analysis of the visual representation of the rendered page content, and using that structure for improving ranking algorithms in a search engine.\\
(development halted after 4 months due to time constraints)}
	 {
	 	\href{http://nodejs.org}{Node.js} -- 
		\href{http://phantomjs.org}{PhantomJS} -- 
		\href{http://www.mongodb.org}{MongoDB}
	 }


\vspace{-3pt}


\item{January 2011 -- June 2011}
	 {}
	 {\href{https://github.com/baygan/Baygan}{Baygan Database}}
	 {An extendable and clearly-modulated framework for introducing students to the intricacies of relational database design. Inspired by \href{http://www.stanford.edu/class/cs140/projects/pintos/pintos.html}{pintos}.\\
(development halted after 6 months due to time constraints)}
	 {
	 	\href{http://www.oracle.com/technetwork/java/}{Java} -- 
		\href{http://www.antlr.org}{ANTLR}
	 }
	 

\vspace{5pt}
	 
	 
\item{June and July 2010}
	 {}
	 {Embedded Search Engine}
	 {A complete, single-purpose search engine (written from scratch), designed to use minimal RAM (60MB) for indexing and searching the English Wikipedia, as the final project for the "Information Retrieval" course.}
	 {
		 \href{http://www.oracle.com/technetwork/java/}{Java}
	 }



\item{June 2012}
	 {}
	 {\href{https://github.com/pooriaazimi/Mini-Java}{MiniJava Parser}}
	 {A parser, complete with type checking, simple static analysis (of variable and function names in their scope), and an informative web-based UI, for the contrived \href{http://www.cambridge.org/resources/052182060X/MCIIJ2e/grammar.htm}{MiniJava} language, as the final project for the "Compiler Design" course.}
	 {
		 \href{http://coffeescript.org}{CoffeeScript} -- 
		 \href{http://zaach.github.io/jison/}{jison} -- 
		 \href{http://d3js.org}{d3.js}
	 }



\item{January 2012}
	 {}
	 {\href{https://github.com/pooriaazimi/secure_file_system}{Secure File System}}
	 {A "secure" web-based storage solution (i.e., all the encryption happens in the browser), with multiple user support, as the final project for the "Information Security" course.}
	 {
		 \href{http://php.net}{PHP} -- 
		 \href{http://www.mongodb.org}{MongoDB}
	 }



\item{April 2012}
	 {}
	 {\href{http://www.data-mining-cup.de/en/review/dmc-2012/}{13\textsuperscript{th} International Data Mining Cup}}
	 {Our team created a bidding agent in Java for the "online" task (ranked 2nd in the Cup), and used a combination of statistical models, neural networks and SVMs for predicting the results of the "offline" task (ranked 13th).}
	 {
	 	\href{http://www.mathworks.com/products/matlab/}{MATLAB} -- 
	 	\href{http://www.oracle.com/technetwork/java/}{Java}
	 }



\item{May 2013}
	 {}
	 {\href{https://github.com/pooriaazimi/twitter}{A (simple) Twitter clone}}
	 {A simple, but fully-featured Twitter clone (with users, tweets, timeline view, following, and an admin interface), as a learning exercise for Ruby on Rails web framework.}
	 {
%		 \href{https://www.ruby-lang.org/en/}{Ruby} -- 
		 \href{http://rubyonrails.org}{Ruby on Rails} -- 
		 \href{http://www.postgresql.org}{PostgreSQL} -- 
		 \href{http://coffeescript.org}{CoffeeScript} -- 
		 \href{http://sass-lang.com}{Sass}
	 }



\item{February 2013}
	 {}
	 {\href{https://github.com/pooriaazimi/adserver}{Ad Server}}
	 {A simple ad server (for tracking ad impressions), as a learning experience for Sinatra web framework.}
	 {
	 	\href{https://www.ruby-lang.org/en/}{Ruby} -- 
		\href{http://www.sinatrarb.com}{Sinatra} -- 
		\href{http://www.sqlite.org}{SQLite}
	 }



\item{July 2012}
	 {}
	 {University Registration System}
	 {A complete and realistic university registration system (server- and client-side), taking into account virtually all the intricacies of registration process, as the final project for "Database Design" lab.}
	 {
	 	\href{http://www.microsoft.com/en-us/sqlserver/default.aspx}{Microsoft SQL Server} -- 
		\href{https://en.wikipedia.org/wiki/C_Sharp_(programming_language)}{C\#}
	 }
	 
	 
	 
\item{2011 -- 2013}
	 {}
	 {OS X and iOS Apps}
	 {Multiple (mostly small) OS X and iOS applications, most notably \href{http://pooriaazimi.github.io/BetterDictionary/}{BetterDictionary} and \href{http://www.turnedondigital.com/?portfolio=farhang-iphone-app}{Farhang}.}
	 {
	 	\href{https://en.wikipedia.org/wiki/Objective-C}{Objective-C} -- 
		\href{https://en.wikipedia.org/wiki/Cocoa_(API)}{Cocoa} -- 
		\href{https://en.wikipedia.org/wiki/Core_Data}{Core Data}
	 }
	 


\item{2010 -- 2013}
	 {}
	 {Open Source Contributions}
	 {Contributing to multiple Open Source projects (code, documentation, and IRC support), including 
	 \href{https://github.com/ggreer/the_silver_searcher}{Ag (the silver searcher)}, 
	 \href{http://bitbucket.org/sjl/hg-prompt/}{hg-prompt}, \href{https://github.com/allending/Kiwi}{Kiwi}, 
	 \href{https://github.com/mxcl/homebrew}{Homebrew}, and 
	 \href{http://fishshell.com}{fish shell}.}
	 {}


\end{project_list}




\vspace{-20pt}




\section{Awards}

\begin{factlist}
\item{2\textsuperscript{nd} place}{\href{http://www.data-mining-cup.de/en/review/dmc-2012/}{13\textsuperscript{th} International Data Mining Cup, Berlin, Germany, 2012}}
\end{factlist}




\newpage


\section{Online Education}

In addition to my normal classes, I have watched the videos, and finished the assignments of, a dozen freely available online courses, including the following CS-related courses:

\begin{itemize}

\item MIT's legendary \href{http://groups.csail.mit.edu/mac/classes/6.001/abelson-sussman-lectures/}{Structure and Interpretation of Computer Programs} {\it(1986 -- by \href{https://en.wikipedia.org/wiki/Hal_Abelson}{Harold Abelson} and \href{https://en.wikipedia.org/wiki/Gerald_Jay_Sussman}{Gerald Jay Sussman})}

\item UC Berkley's \href{http://www.cs.berkeley.edu/~kubitron/courses/cs162-F08/}{Operating Systems and Systems Programming} {\it(2008)}

\item Harvard's \href{http://cs50.tv/2010/fall/}{Introduction to Computer Science} {\it(2010)} and \href{http://cs75.tv/2010/fall/}{Building Dynamic Websites} {\it(2010)}

\item Stanford's \href{http://see.stanford.edu/see/courseinfo.aspx?coll=824a47e1-135f-4508-a5aa-866adcae1111}{Programming Methodology} {\it(2007)}, \href{http://see.stanford.edu/see/courseinfo.aspx?coll=11f4f422-5670-4b4c-889c-008262e09e4e}{Programming Abstractions} {\it(2008)}, \href{http://see.stanford.edu/see/courseinfo.aspx?coll=2d712634-2bf1-4b55-9a3a-ca9d470755ee}{Programming Paradigms} {\it(2008)}, and \href{http://www.stanford.edu/class/cs193p}{iPhone Application Programming} {\it(2013)}
	
\end{itemize}

I'm also taking the following Coursera courses this semester: 

\begin{itemize}

\item EPFL's \href{https://www.coursera.org/course/progfun}{Functional Programming Principles in Scala} {\it(2013 -- by \href{https://en.wikipedia.org/wiki/Martin_Odersky}{Martin Odersky})}

\item EPFL/Typesafe Inc.'s \href{https://www.coursera.org/course/reactive}{Principles of Reactive Programming} {\it(2013 -- by \href{https://en.wikipedia.org/wiki/Martin_Odersky}{Martin Odersky})}

\end{itemize}







\end{document}
